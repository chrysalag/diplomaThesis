\chapter[Εισαγωγή]{Εισαγωγή}
\label{Introduction}
Στην παρούσα διπλωματική εργασία έγινε μια επισκόπιση των αλγορίθμων συνεργατικής διήθησης και του προβλήματος της κρύας εκκίνησης. \par
Υλοποιήθηκε ο αλγόριθμος {\en HIR} \cite{Nikolakopoulos2015126}, o οποίος επιλέχθηκε λόγω της καλής συμπεριφοράς του στο πρόβλημα της κρύας εκκίνησης, η οποία προκύπτει από τη θεώρησή του πάνω στο πρόβλημα αυτό. Η υλοποίηση έγινε σε {\en Java} στο λογισμικό {\en LensKit} \cite{Ekstrand:2011:RRR:2043932.2043958}, λόγω των δυνατοτήτων που προσφέρει στην υλοποίηση αλγορίθμων παραγωγής συστάσεων και στο χειρισμό των απαιτούμενων δεδομένων. \par
Στο 2ο κεφάλαιο αναφέρονται οι διάφορες προσεγγίσεις στον τομέα των συστημάτων παραγωγής συστάσεων και κύρια αυτή της συνεργατικής διήθησης. Ακόμα, εξετάζεται το πρόβλημα της κρύας εκκίνησης.\par 
Στο 3ο κεφάλαιο παρουσιάζεται ο αλγόριθμος {\en HIR}, αναλύεται το μοντέλο που προτείνει και αναφέρεται η πειραματική του αξιολόγηση, η οποία αποδεικνύει την υπεροχή του στο πρόβλημα της κρύας εκκίνησης. \par
Στο 4ο κεφάλαιο γίνεται μια επισκόπιση του {\en LensKit}, της στρατηγικής που ακολουθεί και των πλεονεκτημάτων που προσφέρει. \par
Στο 5ο κεφάλαιο αναλύεται η υλοποίηση του αλγορίθμου {\en HIR}, του {\en Demo} που κάνει χρήση αυτού και των σεναρίων που γράφτηκαν για τον έλεγχο της υλοποίησης. \par
Στο 6ο κεφάλαιο περιέχονται τα συμπεράσματα που προέκυψαν κατά την εκπόνηση της παρούσας 
διπλωματικής εργασίας. \par
Στο 7ο κεφάλαιο παρατίθεται το σύνολο του κώδικα της υλοποίησης, του {\en Demo}, των σεναρίων ελέχγου και του βοηθητικού κώδικα που γράφτηκε για το χειρισμό των συνόλων δεδομένων και για την παραγωγή των απαιτούμενων για τα σενάρια ελέγχου δεδομένων. \par
Στο 8ο και 9ο κεφάλαιο καταγράφονται η ορολογία και τα αρκτικόλεξα που χρησιμοποιήθηκαν στο κείμενο της διπλωματικής. \par
Τέλος, παρατίθεται η σχετική βιβλιογραφία. 