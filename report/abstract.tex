\chapter*{Περίληψη}

Τα συστήματα παραγωγής συστάσεων είναι εργαλεία λογισμικού και τεχνικές που στοχεύουν στο να βοηθήσουν τους χρήστες να επιλέξουν ανάμεσα σε μια πληθώρα διαθέσιμων ειδών. Στην παρούσα διπλωματική εργασία έγινε μια επισκόπηση των διαφόρων προσεγγίσεων στον τομέα των συστημάτων παραγωγής συστάσεων. Εξετάστηκε κυρίως αυτή της συνεργατικής διήθησης καθώς κρίνεται μία από τις πιο επιτυχημένες. Ιδιαίτερη αναφορά έγινε στο πρόβλημα της κρύας εκκίνησης, αφού αποτελεί πρόκληση για την επιτυχή παραγωγή συστάσεων. \par
Υλοποιήθηκε ο αλγόριθμος {\en Hierarchical Itemspace Rank (HIR)}, o οποίος επιλέχθηκε λόγω της καλής συμπεριφοράς του στο πρόβλημα της κρύας εκκίνησης. Η επιτυχία του στον τομέα αυτό προκύπτει από την αξιοποίηση της εγγενούς ιεραρχικής δομής του χώρου των ειδών, μέσω της οποίας αντιμετωπίζει επιτυχώς τα προβλήματα της αραιότητας και της κρύας εκκίνησης και παράγει ποιοτικές συστάσεις.\par
Εκτός από τον ίδιο τον αλγόριθμο, υλοποιήθηκε και ένα {\en demo}, το οποίο τον χρησιμοποιεί για να προτείνει σε χρήστες ταινίες, χρησιμοποιώντας ένα ειδικό για το σκοπό αυτό σύνολο δεδομενων. Οι συστάσεις γίνονται βάσει των βαθμολογιών που έχουν ήδη δώσει οι χρήστες σε κάποιες ταινίες. \par
Η υλοποίηση έγινε σε {\en Java} στο λογισμικό {\en LensKit}. To {\en LensKit} αποτελεί ένα ολοκληρωμένο σύστημα για την υλοποίηση, τη σύγκριση, την πειραματική αξιολόγηση αλλά και την έρευνα πάνω σε αλγορίθμους παραγωγής συστάσεων. Επιλέχθηκε λόγω των δυνατοτήτων που προσφέρει τόσο στον τομέα της υλοποίησης όσο και στο χειρισμό των απαιτούμενων δεδομένων.\par
Η παρούσα υλοποίηση έχει τη δυνατότητα να ενταχθεί σε μία νέα ενότητα υλοποιημένων αλγορίθμων στο {\en LensKit}, η οποία θα εξειδικεύεται στην αντιμετώπιση του προβλήματος της κρύας εκκίνησης. \\ 

\par ΘΕΜΑΤΙΚΗ ΠΕΡΙΟΧΗ: Συστήματα Παραγωγής Συστάσεων
\par ΛΕΞΕΙΣ ΚΛΕΙΔΙΑ: Συνεργατική Διήθηση, Κρύα Εκκίνηση, {\en HIR, LensKit} 

