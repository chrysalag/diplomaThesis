\chapter{\el Ανάλυση της Υλοποίησης}
\label{Chapter5}
Στο κεφάλαιο αυτό παρουσιάζεται η υλοποίηση του αλγορίθμου {\en HIR}. Η υλοποί\-ηση πραγματοποιήθηκε και ελέγχθηκε σε κλώνο του {\en master branch} του {\en LensKit}, όπως αυτό υπάρχει στο {\en GitHub}. Στη συνέχεια, η υλοποίηση, εκτός των ελέγχων, ενσωματώθηκε στο {\en Lenskit Hello project} που παρέχει το {\en LensKit} και περιέχει {\en Demo}, μέσω του οποίου εκτελείται ο επιλεγμένος παραγωγός συστάσεων παίρνοντας ορίσματα από τη γραμμή εντολών. Το {\en Demo} αυτό τροποποιήθηκε, ώστε να διαμορφώνει και να χρησιμοποιεί την υλοποίηση του αλγορίθμου {\en HIR}.
\section{\en HIR Item Scorer}
Κεντρικό τμήμα κάθε παραγωγού συστάσεων είναι ο \en Item Scorer. \el Στο \en LensKit, \el υλοποιείται μέσω της μεθόδου:
\en \begin{footnotesize}
\begin{verbatim}
ResultMap scoreWithDetails(long user, @Nonnull Collection<Long> items)
\end{verbatim}
\end{footnotesize}
\el η οποία δέχεται ως όρισμα το αναγνωριστικό του χρήστη και μια συλλογή από τα είδη από τα οποία καλείται να παράξει συστάσεις. \par
Μέσω ενός αντικειμένου για την πρόσβαση στα γεγονότα που έχει συμμετάσχει ο χρήστης (βαθμολογίες), του \en UserEventDAO, \el παράγεται το ιστορικό του χρήστη, μια περίληψη του οποίου αποθηκεύεται σε ένα \en Sparse Vector. \el Στη συνέχεια, δημιουργείται το διάνυσμα προτίμησης του χρήστη, το οποίο αντιστοιχεί τα αναγνωριστικά του συνόλου των είδων με τη βαθμολογία τους από το χρήστη. Τα αναγνωριστικά όσων ειδών δεν έχουν βαθμολογηθεί από τον χρήστη, αντιστοιχίζονται με μηδέν. Το διάνυσμα αυτό κανονικοποιείται, ώστε τα στοιχεία του να αθροίζουν στη μονάδα. \par
Έπειτα, εφαρμόζεται το κύριο μέρος του αλγορίθμου:
\begin{itemize}
\item Κατασκευάζεται το διάνυσμα κατάταξης του χρήστη, το οποίο αποτελεί αντί\-γραφο του διανύσματος προτίμησής του πολλαπλασιασμένο με 0.1.\\
\item Στη συνέχεια προσπελαύνεται το σύνολο των μη μηδενικών τιμών του, οι οποίες αντιστοιχούν στα είδη για τα οποία υπάρχει πληροφορία για την αλληλεπίδραση του χρήστη με αυτά. Για αυτά τα είδη, καλείται:
\begin{itemize}
\item η συνάρτηση που παρέχει το {\en HIR Model} και επιστρέφει το κανονικοποιημένο διάνυσμα που περιέχει την πληροφορία για τον αριθμό των κοινών βαθμολογίων του με άλλα είδη και
\item η συνάρτηση του {\en HIR Model} που επιστρέφει το κανονικοποιημένο διά\-νυσμα γειτνίασης του τρέχοντος είδους με τα υπόλοιπα βάσει των κατηγοριών στα οποία ανήκει. 
\end{itemize}
\item Τα παραπάνω διανύσματα αποθηκεύονται σε {\en Sparse Vectors} και πολλαπλασιάζονται με την παράμετρο άμεσης συσχέτισης και γειτνίασης αντίστοιχα. Ο γραμμικός συνδυασμός τους πολλαπλασιάζεται με τη βαθμολογία του είδους που εξετάζεται και στη συνέχεια προστίθεται με το διάνυσμα κατάταξης, όπως αυτό έχει τροποποιηθεί, υπολογίζοντας με αυτό τον τρόπο το τελικό διάνυσμα κατάταξης του χρήστη. 
\item Τέλος, το διάνυσμα αυτό προσπελαύνεται και τα στοιχεία του που δεν αντι\-προσωπεύουν είδη τα οποία έχει δει ο χρήστης επιστρέφονται ως αποτέλεσμα της κλήσης του {\en Item Scorer}.
\end{itemize}
\section{\en HIR Model}
Στο {\en HIR Model} παρέχονται από το {\en HIR Model Builder} ως δεδομένα το μητρώο κοινών βαθμολογιών και τα μητρώα που αποτελούν την παραγοντοποίηση του μητρώου γειτνίασης. \par
Έχει δύο συναρτήσεις:
\begin{scriptsize}
\en \begin{verbatim}
public MutableSparseVector getCoratingsVector(long item, Collection<Long> items) 
public MutableSparseVector getProximityVector(long item, Collection<Long> items)
\end{verbatim}
\end{scriptsize}
\el Η πρώτη ανακτά από το μητρώο άμεσης συσχέτισης το (κανονικοποιημένο) {\en Real Vector} που περιέχει τις κοινές του βαθμολογίες και αποθηκεύει τα περιεχόμενα του σε ένα {\en Sparse Vector}, το οποίο και επιστρέφει στον {\en HIR Item Scorer}.\par
Η δεύτερη ανακτά από το μητρώο που αποτελεί τον πρώτο παράγοντα του μητρώου γειτνίασης, το (κανονικοποιημένο) διάνυσμα γραμμή που αντιστοιχεί στο είδος που προσπελαύνεται σε μορφή {\en Real Vector}. Στη συνέχεια το πολλαπλασιά\-ζει από δεξιά με το (κανονικοποιημένο κατά γραμμές) μητρώο που αποτελεί το δεύτερο παράγοντα του μητρώου γειτνίασης. Το αποτέλεσμα αποτελεί το διάνυσμα γειτνίασης, το οποίο αποθηκεύεται σε {\en Real Vector} και τα περιεχόμενά του αντιγράφονται στο {\en Sparse Vector}, το οποίο επιστρέφεται στον {\en HIR Item Scorer}.
\section{\en HIR Model Builder}
Το {\en HIR Model Builder} καλεί τις κλάσεις που παράγουν τα μητρώα του μοντέ\-λου, τα οποία παρέχει στο {\en HIR Model}, το οποίο με τη σειρά του μετά από τους απαιραίτητους υπολογισμούς επιστρέφει στον {\en HIR Item Scorer} τα διανύσματα που χρειάζεται κάθε φορά. Επιπλέον, κατασκευάζει τα διανύσματα των ειδών που περιέχουν τις βαθμολογίες τους, τα οποία χρειάζονται για τον υπολογισμό των κοινών τους βαθμολογιών. 
\section{\en Direct Association Matrix}
Κατασκευάζει ένα {\en Real Matrix}, το οποίο σε κάθε θέση περιέχει τον αριθμό των κοινών βαθμολογιών των ειδών που βρίσκονται στην αντίστοιχη γραμμή και στήλη του. Στη συνέχεια το κανονικοποιεί κατά γραμμές. Το μητρώο αποκτά τα δεδομένα του μέσω της συνάρτησης:
\en \begin{scriptsize}
\begin{verbatim}
public void putItemPair(long id1, SparseVector itemVec1, long id2, SparseVector itemVec2)
\end{verbatim}
\end{scriptsize}
\el και κανονικοποιείται και επιστρέφεται μέσω της:
\en \begin{scriptsize}
\begin{verbatim}
public RealMatrix buildMatrix()
\end{verbatim}
\end{scriptsize}
\section{\en Row Stochastic Factor of Proximity}
\el Η {\en Row Stochastic Factor of Proximity} αποθηκεύει σε ένα {\en Real Matrix} τα δεδομένα που αφορούν τις κατηγορίες που ανήκουν τα είδη (το μητρώο $\mathbf{A}$ του μοντέλου). Αποκτά πρόσβαση στην απαραίτητη πληροφορία μέσω {\en Data Access Object}, του {\en MapItemGenreDAO}. \par
Έπειτα μέσω της μεθόδου:
\en \begin{verbatim}
public RealMatrix RowStochastic()
\end{verbatim}
\el το κανονικοποιεί κατά γραμμές και το επιστρέφει. Για το άθροισμα κατά γραμμές το {\en LensKit} δεν παρέχει συνάρτηση αθροίσματος, αλλά συνάρτηση για τον υπολογισμό της {\en L1} νόρμας διανύσματος. Δεδομένου ότι τα στοιχεία του μητρώου είναι μη αρνητικά, αυτή η συνάρτηση υπολογίζει το άθροισμα κατά γραμμές. 
\section{\en Transposed Factor of Proximity}
Εφαρμόζει την παραπάνω διαδικασία στο $\mathbf{A}^\text{T}$.
\section{Παράμετροι}
Για τις παραμέτρους άμεσης συσχέτισης και γειτνίασης, υλοποιήθηκαν δύο διεπαφές, με τις οποίες επικοινωνεί ο {\en HIR Item Scorer} και λαμβάνει τις προεπιλεγμένες τους τιμές. Επιλέχθηκαν ως τέτοιες το $0,6$ για την άμεσης συσχέτισης και το $0,3$ για τη γειτνίασης, επειδή για αυτές ο {\en HIR} έχει καλύτερη απόδοση. \cite{Nikolakopoulos2015126}
\section{Χειρισμός Δεδομένων}
Για την πρόσβαση στα δεδομένα που αφορούν τις κατηγορίες των ειδών, δημιουργήθηκε η διεπαφή {\en ItemGenreDAO}, η οποία περιέχει τον ορισμό των μεθόδων:
\en \begin{verbatim}
RealVector getItemGenre(long item)
int getGenreSize()
\end{verbatim}
\el οι οποίες χρησιμοποιούνται για την πρόσβαση στα διανύσματα που περιέχουν τις κατηγορίες στις οποίες ανήκουν τα είδη και των αριθμό των κατηγοριών αντίστοιχα. \par
Υλοποίηση της διεπαφής είναι το {\en MapItemGenreDAO}, το οποίο χρησιμοποιεί\-ται για το χειρισμό ενός {\en CSV} αρχείου, το οποίο περιέχει την απαραίτητη πληροφορία. Γίνεται η θεώρηση ότι αυτό το {\en CSV} περιέχει στην πρώτη θέση το {\en id} του είδους, στη δεύτερη το όνομά του και στην τρίτη ένα διάνυσμα, το οποίο έχει 1 αν το είδος ανήκει στην κατηγορία που αντιπροσωπεύει αυτή η θέση, αλλιώς 0. Τα 0 και 1 είναι χωρισμένα με «$|$».
\section{Έλεγχος της υλοποίησης}
Μέσω {\en test} αρχείων έγινε έλεγχος της λειτουργίας των:
\begin{itemize}
\item {\en MapItemGenreDAO}
\item {\en HIRItemRecommender}
\item {\en HIRModelBuilder}
\item {\en HIRItemScorer}
\end{itemize}\par
Για τον έλεγχο του {\en MapItemGenreDAO}, θεωρήθηκε ένα μικρό σύνολο δεδομένων 6 ταινιών και 20 κατηγοριών, το οποίο είναι στη μορφή που αναφέρθηκε παραπάνω. Στη συνέχεια, ελέγχθηκε αν επιστρέφονται δεδομένα για ταινίες που δεν υπάρχουν πραγματικά στο σύνολο δεδομένων, ο σωστός αριθμός κατηγοριών, τα διανύσματα όπως έχουν οριστεί στο σύνολο δεδομένων και αν εντοπίζονται όλα τα αναγνωριστικά των ειδών του συνόλου δεδομένων. \par
Για τον {\en HIRItemRecommender}, ελέγχθηκε ότι διαμορφώνεται σωστά ο παραγωγός συστάσεων καλώντας όλα τα απαραίτητα για την υλοποίησή του τμήματα.\par
Για τον {\en HIRModelBuilder} θεωρήθηκε το ίδιο σύνολο δεδομένων ταινιών και δύο μικρά σύνολα βαθμολογιών και ελέγθηκε ότι επιστρέφονται τα σωστά διανύσματα άμεσης συσχέτισης και γειτνίασης. \par
Για τον {\en HIRItemScorer}, χρησιμοποιήθηκαν οι ίδιες ταινίες και ένα μικρό σύνολο δεδομένων τριών χρηστών που είχαν βαθμολογήσει τις ίδιες τρεις ταινίες με διαφορετικό τρόπο ο καθένας. Ελέχθηκε ότι επιστρέφονται οι βαθμολογίες που προβλέπονται θεωρητικά από τον αλγόριθμο με ακρίβεια $10^{-6}$. Η επιστροφή των σωστών βαθμολογιών διασφαλίζει ότι {\en HIRItemRecommender} θα επιστρέψει τη σωστή τους κατάταξη. \par
Για την παραγωγή των απαραίτητων τιμών ελέγχου για τον {\en HIRModelBuilder} και τον {\en HIRItemScorer} χρησιμοποιήθηκε το {\en Octave}.\par
Τέλος για το χειρισμό των εξαρτήσεων γράφτηκε ένα {\en script} σε {\en Gradle}.
\section{ \en Demo}
Για την υλοποίηση του {\en demo} που χρησιμοποιεί τον παραγωγό συστάσεων του {\en HIR}, γράφτηκε ένα αρχείο διαμόρφωσης σε {\en Groovy}, το οποίο συνδέει τη διεπαφή του {\en Item Scorer} με την υλοποίηση του {\en HIR Item Scorer} και αρχικοποιεί τις παραμέτρους άμεσης συσχέτισης και γειτνίασης. \par
Για τις ανάγκες του {\en Demo}, αλλά και της συνολικής υλοποίησης, απαιτείται:
\begin{itemize}
\item Ένα σύνολο δεδομένων που παρέχει πληροφορίες για τα είδη, τις κατηγορίες που ανήκουν και τις βαθμολογίες των χρηστών για αυτά.
\item Το τμήμα του συνόλου δεδομένων να δίνει πληροφορίες για τις κατηγορίες όπως απαιτείται από το {\en Data Access Object} που τα χειρίζεται.
\item Τα είδη να έχουν συνεχόμενα αναγνωριστικά ({\en 0-index}), αφού οι πληροφορίες για αυτά αποθηκεύονται σε μητρώα και όχι σε {\en maps}.
\end{itemize}\par
Για τις ανάγκες του {\en Demo} χρησιμοποιήθηκε η τελευταία έκδοση (8/2015) του {\en ml-latest-small}\footnote{\en \url{http://grouplens.org/datasets/movielens/latest/}}, το οποίο περιέχει 100.000 βαθμολογίες σε 9.000 ταινίες από 700 χρήστες. Το αρχείο που περιείχε τις ταινίες αντιγράφηκε, ώστε να τροποιηθεί και να περιέχει με κατάλληλο τρόπο τις πληροφορίες για τις κατηγορίες που ανήκουν. Η τροποποίηση του παραπάνω συνόλου δεδομένων βάσει των προϋποθέσεων που περιγράφηκαν παραπάνω έγινε με χρήστη {\en Python script}.\par
Το {\en Demo} διαβάζει τα αρχεία του συνόλου δεδομένων και διαμορφώνει τον {\en HIR Item Scorer} με τη βοήθεια και του {\en Groovy script}. Στη συνέχεια καλεί αυτή τη διαμόρφωση μέσω ενός {\en Item Recommender} και προτείνει από 10 ταινίες για τον χρήστη ή τους χρήστες των οποίων τα αναγνωριστικά λαμβάνει από τη γραμμή εντολών κατά την κλήση του. 
\section{Εργαλεία Υλοποίησης}
Η υλοποίηση έγινε με χρήση της {\en Community} έκδοσης του {\en IntelliJ IDEA 14.1.5}. Η έκδοση της {\en Java} που χρησιμοποιήθηκε είναι η {\en java-1.7.0-openjdk-amd64}, της {\en Python} η 2.7.10 και του {\en Octave} η {\en GNU Octave, 3.8.1}. Το λειτουργικό του υπολογιστή που έγινε η υλοποίηση είναι {\en Ubuntu 14.04.3 LTS}. Ο υπολογιστής που χρησιμοποιήθηκε έχει επερξεγαστή {\en Intel® Core™ i5-5200U CPU @ 2.20GHz × 4} και 8{\en GB RAM.} Η συγγραφή της διπλωματικής έγινε σε \LaTeX\ με τη χρήση {\en TeXstudio}.