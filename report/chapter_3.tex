\chapter{Ο Αλγόριθμος {\en HIR}}
\label{Chapter3}
Η αραιότητα ({\en sparsity}) είναι εγγενές χαρακτηριστικό των συστημάτων παραγωγής συστάσεων και αποτελεί ένα από τα πιο δύσκολα ζητήματα που έχουν να αντιμετωπίσουν οι αλγόριθμοι συνεργατικής δίηθησης. Σε αυτό ακριβώς το πρόβλημα, ακόμα και στην ακραία του μορφή που είναι το πρόβλημα της κρύας εκκίνησης, απαντάει ο {\en Hierarchical Itemspace Rank (HIR). O HIR} εκμεταλλεύεται την ιεραρχική δομή του χώρου των ειδών, ώστε να ξεπεράσει τους περιορισμούς στην ποιοτική παραγωγή συστάσεων\cite{Nikolakopoulos2015126}.\footnote{Κύρια πηγή του παρόντος κεφαλαίου ειναι η δημοσίευση \cite{Nikolakopoulos2015126}}
\section{Εισαγωγή}
Το έναυσμα για τον αλγόριθμο {\en HIR} προέκυψε από τα αποτελέσματα ενός καινοτόμου πλαισίου εργασίας στον Παγκόσμιο Ιστό \cite{Nikolakopoulos:2013:NNR:2433396.2433415}, το οποίο κατέδειξε τα οφέλη της απεικόνισης των έμμεσων σχέσεων μεταξύ των ιστοσελίδων που προκύπτουν από την εγγενή ιεραρχική δομή του Παγκόσμιου Ιστού και της εκμετάλλευσής τους για την καλύτερη κατάταξη των ιστοσελίδων που επιστρέφονται από μία μηχανή αναζήτησης. \par
Ο αλγόριθμος {\en HIR} εκμεταλλεύεται την ιεραρχική δομή που υπάρχει εκ φύσεως στο χώρο των ειδών, ώστε να χαρακτηρίσει την σχέσεις μεταξύ των ειδών σε ένα μακροσκοποπικό επίπεδο. Για το σκοπό αυτό αναλύει το χώρο των ειδών σε σύνολα στενά συνδεδεμένων στοιχείων και χρησιμοποιεί αυτή την ανάλυση για να εκμεταλλευτεί τις έμμεσες σχέσεις που προκύπτουν ανάμεσα στα είδη. \par
Κεντρική του ιδέα είναι ο συνδυασμός των άμεσων σχέσεων που προκύπτουν από τις αλληλεπιδράσεις των χρηστών με τα είδη και τις έμμεσες που προκύπτουν από τις σχέσεις μεταξύ των ειδών με στόχο την αντιμετώπιση του προβλήματος της αραιότητας και τη βελτίωση της ποιότητας των συστάσεων.
\section{Το πλαίσιο εργασίας}
Στην ενότητα αυτή παρουσιάζεται το μοντέλο εργασίας του αλγορίθμου.
\subsection{Σημειογραφία}
\begin{itemize}
 \item Όλα τα διανύσματα-στήλες αναπαριστώνται με έντονα πεζά γράμματα.
 \item Όλα τα μητρώα με έντονα κεφαλαία γράμματα.
 \item $\mathbf{c_j^T}$ είναι η $j$ στήλη του μητρώου $\mathbf{C}$ και $C_{ij}$ είναι το στοιχείο στη γραμμή $i$ και στη στήλη $j$ του μητρώου $\mathbf{C}$.
 \item Με κεφαλαία καλλιγραφικά γράμματα ορίζονται τα σύνολα και
 \item το $\triangleq$ χρησιμοποιείται στους ορισμούς.
\end{itemize}
\subsection{Ορισμός του μοντέλου}
Ορίζονται τα σύνολα:
\begin{itemize}
 \item Χρηστών: $\pazocal{U} = \{u_1, u_2, \cdots, u_n\}$
 \item Ειδών: $\pazocal{V} = \{ v_1, v_2, \cdots, v_m \}$
 \item Βαθμολογιών: $\pazocal{R}$, το οποίο αποτελείται από πλειάδες $t_{ij} = (u_i, v_j, r_{ij})$, όπου $r_{ij}$ είναι η βαθμολογία του χρήστη $u_i$ για το είδος $v_j$, η οποία μπορεί να αντιστοιχεί είτε σε ένα θετικό αριθμό είτε σε μια δυαδική μορφή που θα παρουσιάζει την αλληλεπίδραση ή μη του χρήστη με ένα είδος.
 \item Διαμέρισης του συνόλου $\pazocal{R}$: $\{\pazocal{L}, \pazocal{T}\}$, με το $\pazocal{L}$ να αποτελεί το σύνολο εκπαίδευσης και το $\pazocal{T}$ συνολο ελέγχου. Το $\pazocal{L}_i$ περιέχει τα είδη που έχει βαθμολογήσει ο $u_i$ και ανήκουν στο σύνολο εκπαίδευσης. Το $\pazocal{T}_i$ περιέχει τα είδη που έχει βαθμολογήσει ο $u_i$ και ανήκουν στο σύνολο ελέγχου: $\pazocal{L}_i \triangleq \{v_k : t_{ik} \in \pazocal{L} \}$, $\pazocal{T}_i \triangleq \{v_l : t_{il} \in \pazocal{T} \}$.
\end{itemize}\par
Κάθε χρήστης $u_i$ συσχετίζεται με το διάνυσμα προτίμησής του, $\boldsymbol{\omega}^{i}$, του οποίου τα μη μηδενικά στοιχεία αντιστοιχούν στις βαθμολογίες που έχει δώσει ο χρήστης και ανήκουν στο $\pazocal{L}$ και για τον οποίο ισχύει $\pazocal{L}_i \neq \emptyset$. Το διάνυσμα προτίμησης είναι κανονικοποιημένο, ώστε τα στοιχεία του να αθροίζουν στη μονάδα. 
\begin{center}
$\boldsymbol{\omega}^i = [\omega_1^i, \omega_2^i, \cdots, \omega_m^i ]$
\end{center}\par
Στόχος της μεθόδου που προτείνει ο αλγόριθμος {\en HIR} είναι η αντιστοίχιση του σε μία προσωποποιημένη κατανομή πάνω στο χώρο των ειδών.\par
Επιπλέον ορίζεται μια οικογένεια μη κενών συνόλων πάνω στο χώρο ειδών $\pazocal{V}$ με βάση κάποιο κριτήριο κατηγοριοποίησης των ειδών:
\begin{center}
$\pazocal{D} \triangleq \{\pazocal{D}_1, \pazocal{D}_2, \cdots, \pazocal{D}_K\}$, με $\pazocal{V} = \bigcup\limits_{k=1}^{K} \pazocal{D}_k$.
\end{center}
Τέλος, ορίζεται το $\pazocal{G}_v$, το οποίο αποτελεί την ένωση των συνόλων που περιέχουν το $v$ και $N_v$ το πλήθος αυτών:
\begin{center}
$\pazocal{G}_v \triangleq \bigcup\limits_{v \in \pazocal{D}_k} \pazocal{D}_k$.
\end{center}
\subsubsection{Μητρώο Άμεσης Συσχέτισης $\mathbf{C}$}
Το μητρώο άμεσης συσχέτισης $\mathbf{C}$ στοχεύει στο να απεικονίσει τις άμεσες σχέσεις που προκύπτουν ανάμεσα στα στοιχεία του συνόλου των ειδών $\pazocal{V}$. Κάθε στοιχείο του $\pazocal{V}$ συσχετίζεται με μία διακριτή κατανομή πάνω σε αυτό, η οποία ποσοτικοποιεί τις ομοιότητες των στοιχείων του. 
\begin{center}
$\mathbf{c_\mathbf{v}^T} = [c_1, c_2, \cdots, c_m]$
\end{center}
Για να οριστεί το μητρώο $\mathbf{C}$, χρειάζεται να οριστεί η οικογένεια συνόλων $\pazocal{U}_{ij} \subseteq \pazocal{U}$, που αντιπροσωπεύει το σύνολο των χρηστών που έχουν βαθμολογήσει και το $v_i$ και το $v_j$. 
\begin{center}
$\pazocal{U}_{ij} \triangleq \left\{
    \begin{array}{@{} l c @{}}
      u_k : (v_i \in \pazocal{L}_k) \wedge (v_j \in \pazocal{L}_k) & i \neq j \\
      \emptyset & i = j
    \end{array}\right.$
\end{center}
Ορίζεται τώρα το μητρώο $\mathbf{Q}$, με $\mathbf{Q}_{ij} \triangleq |\pazocal{U}_{ij}|$. Το μητρώο αυτό είναι συμμετρικό και η διαγώνιός του ειναι μηδενική. Η στοχαστική έκδοση $\mathbf{\hat{Q}}$ αντιστοιχεί στο γράφημα συσχέτισης των ειδών, όπως ορίζεται στο \cite{Gori:2007:IRB:1625275.1625720}, όπου κάθε ακμή του γραφήματος έχει ως βάρος το αντίστοιχο στοιχείο του μητρώου. Το μητρώο αυτό σε αντίθεση με το $\mathbf{Q}$ δεν είναι συμμετρικό. Το μητρώο $\mathbf{C}$ ορίζεται ως:
\begin{center}
$\mathbf{C} \triangleq \mathbf{\hat{Q}} + \frac{1}{m}\mathbf{a}\mathbf{e}^T$
\end{center}
όπου το διάνυσμα $\mathbf{a} \in \Re^m$ έχει μονάδες στα στοιχεία που αντιστοιχούν στις μηδενικές γραμμές του $\mathbf{Q}$ και το $\mathbf{e}^T \in \Re^m$ είναι το μοναδιαίο διάνυσμα. Το μητρώο που προστίθεται στο $\mathbf{\hat{Q}}$ στοχεύει στο να αντικαταστήσει τις μηδενικές γραμμές του με μια ομοιόμορφη κατανομή στο χώρο των ειδών. Αυτή η μετατροπή συμβάλλει στο να προκύψει κάποια πληροφορία για ζεύγη ειδών όπου τα σύνολα των χρηστών που τις έχουν βαθμολογήσει είναι ξένα μεταξύ τους. Τέτοια σύνολα είναι πιο πιθανό να βρεθούν σε συστήματα συστάσεων που αντιμετωπίσουν το πρόβλημα της κρύας εκκίνησης. 
\subsubsection{Μητρώο Έμμεσης Συσχέτισης $\mathbf{D}$}
Το μητρώο $\mathbf{D}$ στοχεύει στο να απεικονίσει τις έμμεσες σχέσεις ανάμεσα στο χώρο των ειδών όπως προκύπτουν από την ιεραρχική του δομή. Αυτή η δομή προκύπτει από το γεγονός ότι η έκφραση προτίμησης ενός χρήστη για ένα είδος υποδηλώνει το ενδιαφέρον του για τις κατηγορίες στις οποίες ανήκει και κατ'επέκταση και για τα είδη που ανήκουν σε αυτές. Η απεικόνιση αυτών των σχέσεων μπορεί να φανεί πολύ χρήσιμη για την αντιμετώπιση του προβλήματος της αραιότητας. \par
Με βάση τα παραπάνω, κάθε γραμμή του $\mathbf{D}$ συσχετίζεται με ένα διάνυσμα πιθανότητας $\mathbf{d}_v^T$, το οποίο κατανέμει ομοιόμορφα τη μάζα του στα $N_v$ διαφορετικά σύνολα του $\pazocal{D}$ που αποτελούν το $\mathbf{G}_v$ και στη συνέχεια στα είδη που αποτελούν το καθέ ένα από αυτά. Κάθε στοιχείο του $\mathbf{D}$, ορίζεται ως:
\begin{center}
$\mathbf{D}_{ij} \triangleq \sum\limits_{\pazocal{D}_k \in \pazocal{G}_{v_i}, v_j \in \pazocal{D}_k} \frac{1}{N_{v_i}|\pazocal{D}_k|}$
\end{center}
Τόσο από τον ορισμό του, όσο και από τον ορισμό της οικογένειας συνόλων $\mathbf{D}$, προκύπτει ότι το $\mathbf{D}$ είναι στοχαστικό κατά γραμμές. 
\subsection{Ο Αλγόριθμος Ιεραρχικής Κατάταξης βάσει του χώρου των ειδών}
Με βάση τα παραπάνω, μπορεί να οριστεί το προσωποποιημένο διάνυσμα κατάταξης του {\en HIR}, ως η κατανομή πιθανότητας πάνω στο χώρο των ειδών, όπως παράγεται από τον αλγόριθμο 1:
\begin{algorithm}[ht]
  \caption{Ιεραρχική Κατάταξη Βάσει του Χώρου των Ειδών}\label{}
  \begin{algorithmic}[1]
    \Require Μητρώα $\mathbf{C}$, $\mathbf{D} \in \Re^{m\times m}$, παράμετροι $\alpha$, $\beta$ : $\alpha, \beta > 0$, $\alpha + \beta < 1$ και το προσωποποιημένο διάνυσμα προτίμησης $\boldsymbol{\omega}\in \Re^m$
    \Ensure Το διάνυσμα κατάταξης για τον χρήστη, $\boldsymbol{\pi}\in \Re^m$
    \medskip
    \State $\boldsymbol{\pi}^T \gets (1-\alpha - \beta)\boldsymbol{\omega}^T$
    \ForAll{$\omega_j \neq 0$} 
      \State $\boldsymbol{\pi}^T \gets \boldsymbol{\pi}^T + \omega_j(\alpha\mathbf{c}_{j}^{T} + \beta\mathbf{d}_{j}^{T})$
    \EndFor
    \State \Return $ \boldsymbol{\pi} $
  \end{algorithmic}
\end{algorithm}
\begin{theorem}
Για κάθε διάνυσμα προτίμησης $\boldsymbol{\omega}$, το προσωποποιημένο διανυσμα $\boldsymbol{\pi}$, το οποίο παράγεται από τον αλγόριθμο \en HIR \el είναι ένα καλώς ορισμένο κανονικοποιημένο διάνυσμα συστάσεων, το οποίο απεικονίζει μία κατανομή πιθανότητας στο σύνολο των ειδών $\pazocal{V}$.
\end{theorem}
\begin{proof}
Για κάθε διάνυσμα προτίμησης $\boldsymbol{\omega}$ (το οποίο είναι από τον ορισμό του μη αρνητικό) και για $\alpha, \beta > 0$, $\alpha+\beta<1$, το $\boldsymbol{\pi}$ είναι μη αρνητικό διάνυσμα. Επομένως, αρκεί να δειχθεί ότι $\boldsymbol{\pi}^{\text{T}}\mathbf{e} = 1$:
\begin{center}
$\boldsymbol{\pi}^{\text{T}}\mathbf{e} = \{ (1 - \alpha - \beta) \boldsymbol{\omega}^{\text{T}} + \alpha \sum\limits_{j:\boldsymbol{\omega}_j\neq 0} \boldsymbol{\omega}_j\mathbf{c}_j^{\text{T}} + \beta \sum\limits_{j:\boldsymbol{\omega}_j\neq 0} \boldsymbol{\omega}_j\mathbf{d}_j^{\text{T}}\}\mathbf{e}$
\end{center}
Δεδομένου ότι τα στοιχεία του διανύσματος $\boldsymbol{\omega}$ είναι κανονικοποιημένα να αθροί\-ζουν στη μονάδα και τα μητρώα $\mathbf{C}$ και $\mathbf{D}$ είναι στοχαστικά κατά γραμμές, προκύπτει ότι: 
\begin{center}
$\boldsymbol{\pi}^{\text{T}}\mathbf{e} = (1 - \alpha -\beta) + 
\alpha \sum\limits_{j:\boldsymbol{\omega}_j\neq 0} \boldsymbol{\omega}_j + \beta \sum\limits_{j:\boldsymbol{\omega}_j\neq 0} \boldsymbol{\omega}_j = (1 - \alpha -\beta) + \alpha + \beta = 1 $
\end{center}
\end{proof}
\subsection{Ζητήματα Απαιτούμενης Μνήμης}
Το μητρώο $\mathbf{C}$ είναι εγγενώς αραιό και κλιμακώνεται πολύ καλά με την αύξηση του αριθμού των χρηστών. Η αύξηση του αριθμού των χρηστών οδηγεί μόνο σε αύξηση του αριθμού των μη μηδενικών στοιχείων του, αφού η διάστασή του εξαρτάται μόνο απο το πλήθος των ειδών. Σε πραγματικές εφαρμογές αυτό το πλήθος αυξάνει πολυ αργά. \cite{Gori:2007:IRB:1625275.1625720}\par
Το μητρώο $\mathbf{D}$ είναι από τον ορισμό του χαμηλής τάξης για $K<m$. Το $\mathbf{D}$ μπορεί να παραγοντοποιηθεί και να επιτευχθεί έτσι αποδοτικότερη αποθήκευση του και μείωση του υπολογιστικού κοστους. Για το σκοπό αυτό, ορίζεται το μητρώο $\mathbf{A} \in \Re^{m\times K}$, ώστε:
\begin{center}
$  \pazocal{A}_{ik} \triangleq \left\{
    \begin{array}{@{} l c @{}}
      1 & \text{αν } v_i \in \pazocal{D}_k \\
      0 & \text{αλλιώς}
    \end{array}\right.$
\end{center}
Το $\mathbf{D}$ μπορεί να γραφεί ως το γινόμενο των στοχαστικών κατά γραμμές εκδόσεων του $\mathbf{A}$ και του αναστρόφου του, αντίστοιχα.
\begin{center}
$\mathbf{D} = \mathbf{XY}$, $\mathbf{X}\in\Re^{m\times K}$, $\mathbf{Y}\in\Re^{K\times m}$, με \\
$\mathbf{X} \triangleq \mathbf{S}^{-1}\mathbf{A}$ και \\
$\mathbf{Y} \triangleq \mathbf{T}^{-1}\mathbf{A}^\text{T}$
\end{center}
όπου τα μητρώα $\mathbf{S} \triangleq \text{\en diag}(\mathbf{Ae})$ και $\mathbf{T} \triangleq \text{\en diag}(\mathbf{A}^{\text{T}}\mathbf{e})$ είναι διαγώνια.
\begin{lemma}
Τα μητρώα $\mathbf{X}$ και $\mathbf{Y}$ είναι καλώς ορισμένα για οποιαδήποτε παραγοντοποίηση του $\mathbf{D}$ ικανοποιώντας τον ορισμό του. 
\end{lemma}
\begin{proof}
Αρκεί να δειχθεί ότι τα $\mathbf{S}$ και $\mathbf{T}$ είναι αντιστρέψιμα για οποιοδήποτε μητρώο $\mathbf{A}$. Από τον ορισμό του $\mathbf{D}$ προκύπτει ότι κάθε γραμμή και κάθε στήλη του $\mathbf{A}$ αντιστοιχεί σε ένα μη μηδενικό διάνυσμα στον $\Re^K$ και $\Re^m$ αντίστοιχα. \par
Η ύπαρξη μηδενικής γραμμής στο μητρώο $\mathbf{A}$, θα σήμαινε ότι το σύνολο $\pazocal{V}$ δεν αποτελεί την ένωση της οικογένειας συνόλων $\pazocal{D}_k$, άτοπο από τον ορισμό του $\pazocal{D}$. \par
Η ύπαρξη μηδενικής στήλης στο μητρώο $\mathbf{A}$, θα προέκυπτε μόνο αν κάποιο από τα $\pazocal{D}_k$ ήταν κενό, το οποίο αντιτίθεται στον ορισμό τους και δε θα είχε νόημα βάσει του ορισμού του μοντέλου.\par
Τα παραπάνω οδηγούν στο συμπέρασμα ότι τα διανύσματα $\mathbf{Ae}$,  $\mathbf{A}^{\text{T}}\mathbf{e}$ είναι αυστηρά θετικά, πράγμα που διασφαλίζει την ανιστρεψιμότητα των μητρώων $\mathbf{S}$, $\mathbf{T}$.
\end{proof}
\subsection{Υπολογιστικά Ζητήματα}
Το κόστος εκτέλεσης του αλγορίθμου 1 είναι μικρό καθώς η δομή επανάληψης απαιτεί $\pazocal{O}(|\pazocal{V}|)$ πράξεις και εκτελείται $\frac{|\pazocal{L}_i|}{m}$ φορές καθώς οι χρήστες αλληλεπιδρούν με ένα πολύ μικρό μέρος των διαθέσιμων ειδών.\par
Η παραγοντοποίηση του $\mathbf{D}$ ως γινόμενο δύο εξαιρετικά αραιών μητρώων, καταργώντας την ανάγκη απευθείας υπολογισμού του, καθώς και το χαμηλής πυκνότητας μητρώο βαθμολογιών, μειώνουν και άλλο το υπολογιστικό κόστος καθώς επιτρέπουν τον μαζικό υπολογισμό των προσωποποιημένων διανυσμάτων κατάταξης που υπολογίζει ο αλγόριθμος 1. \par
Για τον υπολογισμό αυτό χρειάζεται να οριστεί ένα μητρώο $\boldsymbol{\Omega}\in\Re^{n\times m}$, τέτοιο ώστε: 
\begin{center}
$\boldsymbol{\Omega} \triangleq \left[     
\begin{array}{c}
\boldsymbol{(\omega^1)}^\text{T} \\
\boldsymbol{(\omega^2)}^\text{T} \\
\vdots \\ 
\boldsymbol{(\omega^n)}^\text{T}
\end{array}\right]$ 
\end{center}\par
Ο μαζικός αυτός υπολογισμός του {\en HIR} υπολογίζεται με τον Αλγόριθμο 2, με τις γραμμές του μητρώου $\boldsymbol{\Pi}\in \Re^{n\times m}$, να περιέχουν τα διανύσματα συστάσεων για κάθε χρήστη του συνόλου $\pazocal{U}$.
\begin{algorithm}[ht]
  \caption{Μαζικός Υπολογισμός του {\en HIR}}\label{}
  \begin{algorithmic}[1]
    \Require Μητρώα $\mathbf{C}$, $\mathbf{X}$, $\mathbf{Y}$, παράμετροι $\alpha$, $\beta$ : $\alpha, \beta > 0$, $\alpha + \beta < 1$ και το προσωποποιημένο μητρώο προτίμησης $\boldsymbol{\Omega}$
    \Ensure Το μητρώο κατάταξης για τους χρήστες, $\boldsymbol{\Pi}\in \Re^{n\times m}$
    \medskip
    \State $\boldsymbol{\Pi} \gets (1-\alpha - \beta)\boldsymbol{\Omega}$
    \State $\boldsymbol{\Pi} \gets \boldsymbol{\Pi} + \alpha(\boldsymbol{\Omega}\mathbf{C}) + \beta((\boldsymbol{\Omega}\mathbf{X})\mathbf{Y})$
    \State \Return $ \boldsymbol{\Pi} $
  \end{algorithmic}
\end{algorithm}
\section{Πειραματική Αξιολόγηση}
Η πειραματική αξιολόγηση\cite{Nikolakopoulos2015126} του {\en HIR}, έγινε στους τομείς των συστάσεων ταινιών και μουσικής. \par
Για τον πρώτο τομέα, τα πειράματα έγιναν στα σύνολα δεδομένων {\en MovieLens100K} και {\en MovieLens1M}, με το διαχωρισμό των ταινιών σε κατηγορίες να αποτελεί το κριτήριο για τον ορισμό του μητρώου έμμεσης συσχέτισης. Τα σύνολα αυτά χρησιμοποιούνται ευρέως για τον έλεγχο απόδοσης των συστημάτων παραγωγής συστάσεων. Τα σύνολα αυτά περιλαμβάνουν:
\begin{center}
\begin{longtable}{|c|c|c|c|}
\hline  
\multicolumn{1}{|c|}{\textbf{\el Σύνολα Δεδομένων}} &
\multicolumn{1}{|c|}{\textbf{\el Χρήστες}} &
\multicolumn{1}{|c|}{\textbf{\el Είδη}} &
\multicolumn{1}{|c|}{\textbf{\el Βαθμολογίες}}\\\hline
\endfirsthead
\hline 
\en MovieLens100K & 943 & 1682 & 100.000 \\\hline
\en MovieLens1M & 6040 & 3883 & 1.000.209\\\hline
\end{longtable}
\end{center}\par
Για το δεύτερο τομέα, τα πειράματα έγιναν στο σύβολο δεδομένων {\en Yahoo!R2Music} με κριτήριο κατηγοριοποίησης τους καλλιτέχνες που ερμηνεύουν τα τραγούδια. Το σύνολο αυτό περιλαμβάνει:
\begin{center}
\begin{longtable}{|c|c|c|}
\hline  
\multicolumn{1}{|c|}{\textbf{\el Χρήστες}} &
\multicolumn{1}{|c|}{\textbf{\el Είδη}} &
\multicolumn{1}{|c|}{\textbf{\el Βαθμολογίες}}\\\hline
\endfirsthead
\hline 
1.823.179 & 136.736 & 717.872.016\\\hline
\end{longtable}
\end{center}\par
Η σύγκριση του {\en HIR} έγινε με τους ακόλουθους αλγορίθμους παραγωγής συστάσεων που βασίζονται σε κατατάξεις: 
\begin{itemize}
 \item {\en L} και {\en Katz}, οι οποίοι βασίζονται σε ομοιότητα κόμβων
 \item {\en First Passage Time (FP)} και {\en Matrix Forest Algorithm (FMA)} που ακολουθούν την προσέγγιση του τυχαίου περιπάτου
 \item {\en ItemRank} \cite{Gori:2007:IRB:1625275.1625720} και
 \item {\en PureSVD}
\end{itemize}\par
Οι δύο τελευταίοι δε μπορούν να επωφεληθούν από την ιεραρχική δομή του χώρου των ειδών. Η υλοποίηση όλων των αλγορίθμων έγινε σε {\en MATLAB}. \par
Από τους παραπάνω αλγορίθμους μόνο ο {\en ItemRank} και ο {\en HIR} εξαρτώνται αποκλειστικά από το μέγεθος του χώρου των ειδών, του οποίου η διάσταση αυξάνεται πολύ αργά σε πραγματικές εφαρμογές. Σε όλα τα παραπάνω σύνολα δεδομένων ο {\en HIR} ήταν 10-15 φορές πιο γρήγορος. \par
Για τα πειράματα χρησιμοποιήθηκαν οι μετρικές:
\begin{itemize}
 \item {\en Spearman's $\rho$}
 \item {\en Kendall's $\tau$}
 \item {\en Degree of Agreement (DOA)}
 \item {\en Normalized Distance-based Performance Measure}
\end{itemize}\par
H σύγκριση των αλγορίθμων έγινε πάνω στα προβλήματα της Νέας Κοινότητας, Νέων Χρηστών και Νέων Ειδών που αποτελούν τις τρεις 
εκφάνσεις του προβλήματος της κρύας εκκίνησης. \par
Για τη μοντελοποίηση του προβλήματος της Νέας Κοινότητας, φτιάχτηκαν τρία τεχνητά σύνολα δεδομένων που περιείχαν το 10\%, 20\% και 30\% του συνόλου των βαθμολογιών, τα οποία θεωρήθηκε ότι αντιπροσωπεύουν τις αρχικές φάσεις του συστήματος. \par
Στα πειράματα που έγιναν, ο {\en HIR} ανταποκρίθηκε πολύ καλά και αποδείχθηκε ότι παρά το γεγονός ότι οι άμεσες σχέσεις είχαν ελάχιστη συνεισφορά στις αρχές του συστήματος, οι έμμεσες σχέσεις που απεικονίζονται στο μητρώο $\mathbf{D}$ διατηρούνται περισσότερο. Αυτό είχε ως αποτέλεσμα, το διάνυσμα συστάσεων που επιστρέφει ο {\en HIR} να αποδεικνύεται λιγότερο ευαίσθητο στο πρόβλημα αυτό.\par
Για την προσομοίωση του δεύτερου προβλήματος, έγινε τυχαία επιλογή 200 χρηστών που είχαν βαθμολογήσει 100 ή περισσότερα είδη και τυχαία διαγραφή του 96\%, 94\% και 92\% των βαθμολογιών τους, ώστε να κατασκευαστούν τρία σύνολα δεδομένων που να αφορουν νεοεισερχόμενους στο σύστημα χρήστες. \par
Από τις δοκιμές που διεξήχθησαν, για τις διάφορες τιμές του $\beta$ (της μεταβλητής που καθορίζει τη συμβολή του μητρώου έμμεσης συσχέτισης), προέκυψε η θετική συνεισφορά του μητρώου $\mathbf{D}$ στην ποιότητα της κατάταξης των συστάσεων ακόμα και για μικρές τιμές του $\beta$, αλλά και για την περίπτωση που υπήρχε διαθέσιμο μόνο το 4\% των βαθμολογιών των υπό δοκιμή χρηστών. \par
Το αποτέλεσμα αυτό ήταν αναμενόμενο λόγω της θεώρησης του αλγορίθμου {\en HIR} καθώς παρά το ότι δεν είναι γνωστές αρκετές προτιμήσεις των χρηστών, η εκμετάλλευση της έμμεσης πληροφορίας που παρέχεται από τις πρώτες βαθμολογίες τους, δίνει στον {\en HIR} συγκριτικό πλεονέκτημα στην επιτυχή παραγωγή καλών συστάσεων. \par
Για τον έλεγχο της περίπτωσης των νέων ειδών, επιλέχθηκαν τυχαία το 10\%, 12,5\% και 15\% των ειδών που είχαν τουλάχίστον 30 βαθμολογ'οες και αφαιρέθηκε τυχαία το 90\% αυτών για τη δημιουργία 3 συνόλων δεδομένων. \par
Και σε αυτή την περίπτωση η επίδοση του {\en HIR} ήταν πολύ καλή. Στο {\en Yahoo!R2Music} για τις διάφορες μετρικές κατέκτησε από την 1η ως την 3η θέση και για το {\en MovieLens1M} την 1η για όλες τις μετρικές που χρησιμοποιήθηκαν, επιδεικνύοντας σταθερότητα στις κατατάξεις που έδινε και έλλειψη ευαισθησίας στην αραιότητα. \par
Η καλή του επίδοση οφείλεται στη θεώρησή του για τις έμμεσες σχέσεις που προκύπτουν από τις βαθμολογίες των ειδών, δείχνοντας ότι ακόμα και για ανεπαρκή αριθμό βαθμολογιών, τα νέα είδη αντιμετωπίζονται πιο δίκαια. \par
Στα πειράματα που έγιναν, στο {\en MovieLens100K} με τη χρήση κατάλληλων μετρικών φάνηκε ότι ο αλγόριθμος έδινε τις πιο ποιοτικές κατατάξεις από όλους τους αλγορίθμους με τους οποίους συγκρίθηκε. 
\section{Συμπέρασμα}
Συμπερασματικά, ο αλγόριθμος {\en HIR} εκμεταλλεύεται αποτελεσματικά τις σχέσεις των ειδών που προκύπτουν από την κατηγοριοποίησή τους και παράγει ποιοτικές κατατάξεις. Η αραιότητα έχει μειωμένη επίδραση στο μοντέλο που προτάθηκε. Επιπλέον, το μοντέλο αυτό μπορεί να υπολογιστεί αποδοτικά τόσο λόγω του ότι εξαρτάται απόκλειστικά από τη διάσταση του χώρου των ειδών όσο και λόγω των μαθηματικών ιδιοτήτων που διαθέτει. Από την πειραματική αξιολόγηση του αλγορίθμου προέκυψε η καλή συμπεριφορά του απέναντι στο πρόβλημα της κρύας εκκίνησης. \par
Το μοντέλο αυτό είναι επεκτάσιμο και μπορεί να εφαρμοστεί σε περαιτέρω κατηγοριοποίηση του χώρου των ειδών με απεικόνιση των διαφόρων υποκατηγοριών με την εισαγωγή νέων χαμηλής τάξης μητρώων $\mathbf{D}_1$, $\mathbf{D}_2, \dots$ (και αντίστοιχων μεταβλητών $\beta_1$, $\beta_2, \dots$) που θα απεικονίζουν με περισσότερη λεπτομέρεια την ιεραρχία του χώρου. Η εισαγωγή τέτοιων μητρώων δεν επηρεάζει τη διάσταση του μοντέλου. 