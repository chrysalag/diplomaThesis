\chapter{Εισαγωγή στους Aλγορίθμους Συνεργατικής Διήθησης}
\label{chap:theory}
\section{Συστήματα Παραγωγής Συστάσεων}
Τα συστήματα παραγωγής συστάσεων είναι εργαλεία λογισμικού και τεχνικές που στοχεύουν στο να προτείνουν είδη που ενδιαφέρουν τους χρήστες.\cite{ricci2011recommender}\footnote{Κύρια πηγή του παρόντος κεφαλαίου είναι το \cite{ricci2011recommender}.} Προέκυψαν από την ανάγκη διευκόλυνσης των χρηστών να επιλέξουν ανάμεσα από μια πληθώρα διαθέσιμων ειδών.\par
Με δεδομένα τα σύνολα χρηστών, ειδών και άμεσων ή έμμεσων βαθμολογιών των χρηστών για τα είδη, τα συστήματα παραγωγής συστάσεων προσπαθούν είτε να προβλέψουν τις βαθμολογίες των χρηστών για είδη με τα οποία δεν έχουν αλληλεπιδράσει είτε να προτείνουν σε κάποιο χρήστη μία λίστα ειδών που μπορεί να τον ενδιαφέρουν. \cite{Nikolakopoulos2015126} \par
Τα συστήματα παραγωγής συστάσεων διακρίνονται σε έξι κατηγορίες\cite{Burke:2007:HWR:1768197.1768211}:
\begin{enumerate}
 \item Βάσει περιεχομένου ({\en Content-based}): Υπολογίζουν την ομοιότητα των ειδών βάσει των χαρακτηριστικών τους και προτείνουν στο χρήστη παρόμοια είδη με αυτά για τα οποία έχει δείξει ενδιαφέρον.
 \item Συνεργατικής διήθησης ({\en Collaborative filtering}): Bασίζονται στην παρατήρηση ότι οι άνθρωποι συχνά επιλέγουν ανάμεσα στα διάφορα είδη βάσει των συστάσεων που παρέχονται από άλλους.  Θεωρούν ότι αν ένας χρήστης στο παρελθόν είχε παρόμοιες προτιμήσεις με κάποιους άλλους χρήστες, τότε οι συστάσεις που προέρχονται από αυτούς θα τον ενδιαφέρουν και στο μέλλον. Είναι από τις πιο υλοποιημένες προσεγγίσεις \cite{ricci2011recommender} και κρίνεται από τις πιο επιτυχημένες.\cite{Nikolakopoulos2015126}
 \item Δημογραφικά ({\en Demographic}): Προτείνουν με βάση τη χώρα που βρίσκεται ο χρήστης, το φύλο, την ηλικία του και άλλα στοιχεία που γνωρίζουν για αυτόν.
 \item Βάσει γνώσης ({\en Knowledge-based}): Παρέχουν συστάσεις υπολογίζοντας κατά πόσο τα χαρακτηριστικά των ειδών καλύπτουν τις ανάγκες των χρηστών.
 \item Βάσει κοινότητας ({\en Community-based}): Οι συστάσεις που παρέχουν σε κάποι\-ο χρήστη βασίζονται στις προτιμήσεις των φίλων του καθώς αποδεικνύεται ότι οι άνθρωποι τείνουν να εμπιστεύονται περισσότερο τις συστάσεις που προέρχονται από φίλους τους από αυτές παρόμοιων αλλά άγνωστων σε αυτούς χρηστών. \cite{conf/delos/SinhaS01}
 \item Υβριδικά ({\en Hybrid}): Αποτελούν μίξη των παραπάνω κατηγοριών, με τη μία κατηγορία να προσπαθεί να καλύψει τα μειονεκτήματα της άλλης. Για παράδειγμα, τα συστήματα συνεργατικής διήθησης δε μπορούν να προτείνουν είδη για τα οποία δεν έχουν καθόλου βαθμολογίες. Η μίξη ενός τέτοιου συστήματος με ένα που βασίζεται στο περιεχομένο μπορεί να παράξει συστάσεις και για τέτοια είδη. 
\end{enumerate} 
\section{Η Συνεργατική Διήθηση}
Τα συστήματα συνεργατικής δίηθησης δεν αντιμετωπίζουν κάποια από τα μειονεκτήματα των συστήματων που βασίζονται στο περιεχομένο επειδή βασίζονται στις βαθμολογίες των χρηστών. Μπορούν να προτείνουν είδη των οποίων το περιεχομένο δεν είναι γνωστό ή δε μπορεί να ανακτηθεί, αλλά και είδη που ανήκουν σε πολύ διαφορετικές κατηγορίες, αν ο χρήστης έχει δείξει ενδιαφέρον για αυτές. Επιπλέον, βασίζονται στην ποιότητα των ειδών όπως αυτή έχει διαμορφωθεί από τις βαθμολογίες των χρηστών και όχι στο περιεχομένο, το οποίο πολλές φορές δεν αποτελεί κριτήριο για την ποιότητα ενός είδους.\cite{ricci2011recommender} \par 
Οι κύριες κατηγορίες συστήματων συνεργατικής διήθησης είναι αυτή των κο\-ντινότερων γειτόνων ({\en nearest-neighbors}) και η βασιζόμενη στο μοντέλο ({\en model-based}). 
\subsection{Συνεργατική Διήθηση Κοντινότερων Γειτόνων}
Η προσέγγιση που βασίζεται στους κοντινότερους γείτονες είναι αρκετά δημοφιλής λόγω της απλότητας, της αποτελεσματικότητας και της ικανότητάς της να παράγει ακριβείς και προσωποιημένες συστάσεις.\cite{ricci2011recommender} Τα συστήματα που ακολουθούν αυτή την προσέγγιση χρησιμοποιούν απευθείας τις βαθμολογίες των χρηστών για να προβλέψουν τις βαθμολογίες των νέων για εκείνους ειδών. Η χρήση των βαθμολογιών μπορεί να γίνει είτε με τρόπο που βασίζεται στους χρήστες ({\en user-based}) είτε στα είδη ({\en item-based}). Τα πρώτα εκτιμούν το ενδιαφέρον του χρήστη για ένα είδος χρησιμοποιώντας τις βαθμολογίες άλλων χρηστών με τους οποίους έχει βαθμολογήσει με παρόμοιο τρόπο τα κοινά τους είδη (γείτονες). Στα βασιζόμενα στα είδη, η πρόβλεψη για τη βαθμολογία ενός χρήστη γίνεται με βάση το πώς έχει βαθμολογήσει παρόμοια με το υπό εξέταση είδος. Δύο είδη θεωρούνται παρόμοια αν πολλοί χρήστες τα έχουν βαθμολογήσει με παρόμοιο τρόπο. \par
Τα συστήματα αυτά υστερούν όσον αφορά την πρόβλεψη βαθμολογιών σε σχέση με τους καλύτερους αλγορίθμους που βασίζονται στο μοντέλο.\cite{Koren:2008:FMN:1401890.1401944, Takacs:2007:MCG:1345448.1345466} Όμως έχει γίνει πλέον σαφές ότι η ακρίβεια των προβλέψεων δεν είναι ο μόνος παράγοντας που διασφαλίζει την αποτελεσματικότητα του συστήματος. Επιτυχημένη πρόβλεψη δεν είναι μόνο αυτή που προτείνει στο χρήστη απλά ένα νέο είδος, αλλά και αυτή (και αυτό είναι το πιο δύσκολο) που του δίνει την ευκαιρία να ανακαλύψει είδη ή κατηγορίες που μόνος του δε θα το έκανε.\cite{Good:1999:CCF:315149.315352} Δεδομένου ότι αυτά τα συστήματα αξιοποιούν τις σχέσεις μεταξύ των ειδών, είναι πιο πιθανό να προτεί\-νουν σε κάποιο χρήστη κάποιο είδος που δεν έχει συνάφεια με αυτά που έχει ήδη δει, αν έχει λάβει καλή βαθμολογία από ένα γείτονά του. Αυτό δεν εγγυάται την επιτυχία της σύστασης, αλλά μπορεί να βοηθήσει τον χρήστη να διευρύνει τους ορίζοντες του με επιθυμητό για εκείνον τρόπο. \par
Τα κύρια πλεονεκτήματα αυτών των μεθόδων είναι \cite{ricci2011recommender}:
\begin{itemize}
 \item Απλότητα. Είναι πιο εύκολα κατανοητές διαισθητικά και σχετικά πιο απλές στην υλοποίησή τους. 
 \item Δυνατότητα παροχής δικαιολόγησης για τις προβλέψεις που υπολογίζουν. Μπορούν να επιτρέψουν στο χρήστη να καταλάβει πώς προέκυψαν οι συστάσεις προς αυτόν και μπορούν να χρησιμεύσουν σε ένα διαδραστικό σύστημα, όπου οι χρήστες θα μπορούν να επιλέξουν τους γείτονες στους οποίους δίνουν οι ίδιοι σημασία. 
 \item Αποτελεσματικότητα: Δεν απαιτούν κοστοβόρες φάσεις εκπαίδευσης και οι κοντινότεροι γείτονες μπορούν να προϋπολογιστούν για πιο γρήγορες συστάσεις και να αποθηκευτούν με μικρό κόστος στη μνήμη. Αυτό τους επιτρέπει να χρησιμοποιούνται σε εφαρμογές με εκατομμύρια χρήστες και είδη.
 \item Σταθερότητα που οφείλεται στο ότι επηρεάζονται ελάχιστα από την αύξηση των χρηστών, των ειδών και των βαθμολογιών στο σύστημα. Για την παροχή συστάσεων σε νέους χρήστες δε χρειάζεται να ξαναϋπολογιστούν οι ομοιότητες μεταξύ ειδών. Όταν ένα νέο είδος λάβει κάποιες βαθμολογίες, οι μόνες ομοιότητες που υπολογίζονται είναι αυτές που το αφορούν.
\end{itemize}
Η επιλογή ανάμεσα στα δύο είδη συνεργατικής διήθησης βασίζεται στα ακόλουθα κριτήρια\cite{ricci2011recommender}:
\begin{itemize}
 \item Ακρίβεια: Είναι σημαντική η αναλογία χρηστών και ειδών στο σύστημα. Εί\-ναι προτιμότερο να προκύπτουν λιγότεροι γείτονες για τους οποίους όμως μπορεί να υπολογιστεί υψηλής εμπιστοσύνης ομοιότητα.
 \item Αποτελεσματικότητα: Και σε αυτή την περίπτωση έχει σημασία η αναλογία χρηστών και ειδών. Στις περισσότερες περιπτώσεις ο αριθμός των χρηστών υπερβαίνει κατά πολύ αυτό των ειδών και ο υπολογισμός των γειτονικών ειδών είναι προτιμότερος από άποψης απαιτούμενης μνήμης και χρόνου υπολογισμού των ομοιοτήτων. (Ο χρόνος που απαιτείται για την παραγωγή των συστάσεων είναι ο ίδιος.) Σε μεγαλύτερα συστήματα και λαμβάνοντας υπόψη ότι οι χρήστες πρακτικά βαθμολογούν λίγα αντικείμενα, μπορεί να υπάρξει αποδοτική υλοποίηση αν για κάθε χρήστη αποθηκεύονται μόνο οι καλύτερες ομοιότητες. Με τον ίδιο τρόπο δε χρειάζεται να ελέγχονται όλα τα ζεύγη χρηστών ή ειδών.
 \item Σταθερότητα: Για την παροχή ενός σταθερού συστήματος, παίζει ρόλο η συχνότητα και ο ρυθμός αλλαγής του αριθμού των χρηστών. Ο υπολογισμός ομοιοτήτων με βάσει το είδος είναι προτιμότερος σε συστήματα που ο αριθμός τους αυξάνει πιο αργά σε σχέση με αυτόν των χρηστών και αντίστροφα. 
 \item Δυνατότητα παροχής δικαιολόγησης: Οι βασιζόμενες στα είδη μέθοδοι είναι προτιμότερες στις περιπτώσεις που είναι σημαντικό να παρέχεται στο χρήστη και ο λόγος που το σύστημα του προτείνει ένα είδος εκτός από τις περιπτώσεις που ο χρήστης γνωρίζει τους άλλους χρήστες (κοινωνικά δίκτυα). 
 \item Δυνατότητα παροχής απρόσμενων αλλά καλών συστάσεων: Οι μέθοδοι που εξετάζουν την ομοιότητα μεταξύ των χρηστών είναι προτιμότερες σε αυτή την περίπτωση, ιδιαίτερα όταν οι γειτονιές των χρηστών είναι μικρές.
\end{itemize}\par
Το γεγονός ότι σε πραγματικές εφαρμογές οι χρήστες βαθμολογούν λίγα α\-ντικείμενα και το ότι ο υπολογισμός ομοιότητας μεταξύ χρηστών προκύπτει από τις κοινές τους βαθμολογίες οδηγεί την προσέγγιση αυτή να αντιμετωπίζει δυο σημαντικές προκλήσεις: τη μειωμένη κάλυψη και την ευαισθησία στα αραιά δεδομένα. \par
Οι λίγες βαθμολογίες οδηγούν σε λιγότερους γείτονες και έτσι οι μέθοδοι αυτοί δυσκολεύονται να εντοπίσουν ζεύγη χρηστών με παρόμοιες προτιμήσεις, αλλά χωρίς κοινές βαθμολογίες. Αυτό έχει ως αποτέλεσμα λιγότερες συστάσεις, καθώς προτείνονται μόνο είδη που έχουν βαθμολογηθεί από γείτονες, και επομένως μικρότερη κάλυψη του χώρου των ειδών. \par
Η αραιότητα ({\en sparsity}) λόγω των μειωμένων βαθμολογιών είναι πρόβλημα που αντιμετωπίζουν τα περισσότερα συστήματα παραγωγής συστάσεων και στην περίπτωση των συστήματων που βασίζονται στους γείτονες επηρέαζει την ακρίβεια των συστάσεων. Σε συνδυασμό και με την προσθήκη νέων χρηστών και ειδών στο σύστημα, προκύπτει το πρόβλημα της κρύας εκκίνησης ({\en cold-start})\footnote{Στο πρόβλημα αυτό θα γίνει αναφορά στην επόμενη ενότητα.}, το οποίο μπορεί να οδηγήσει σε έλλειψη δικαιοσύνης στο σύστημα.\par
Για την επίλυση των παραπάνω προβλημάτων υπάρχουν δύο αρκετά δημοφιλείς προσεγγίσεις. Η πρώτη βασίζεται στη μείωση της διάστασης των αναπαραστάσεων χρηστών και ειδών, αξιοποιώντας τα πιο σημαντικά χαρακτηριστικά τους. Με αυτόν τον τρόπο μπορεί να ανακαλύψει συσχετίσεις ανάμεσα σε χρήστες και είδη, ακόμα και αν εχουν βαθμολογήσει διαφορετικά είδη ή έχουν βαθμολογηθεί από διαφορετικούς χρήστες. Η δεύτερη εφαρμόζει μεθόδους από τη θεωρία γραφημάτων και μέσω αυτών εντοπίζει μεταβατικές σχέσεις ανάμεσα σε χρήστες και δεδομένα. Έχει επιπλέον τη δυνατότητα να προτείνει και μη αναμενόμενες, αλλά καλές, συστάσεις.
\subsection{Συνεργατική Διήθηση Βασιζόμενη Στο Μοντέλο}
Τα συστήματα αυτής της κατηγορίας χρησιμοποιούν τις βαθμολογίες για να εκπαιδεύσουν ένα μοντέλο πρόβλεψης. Στόχος τους είναι να ανακαλύψουν τα λανθάνοντα χαρακτηριστικά χρηστών και ειδών που βρίσκονται πίσω από τις βαθμολογίες. Σε αυτή την κατηγορία ανήκουν δημοφιλείς μέθοδοι που περιλαμβάνουν μοντέλα που προκύπτουν απο την παραγοντοποίηση του μητρώου βαθμολογιών χρηστών-ειδών. Είναι γνωστές ως μέθοδοι που βασίζονται στον αλγόριθμο {\en SVD (SVD-based)} και διακρίνονται για την ακρίβεια και την ικανότητά τους να ανταποκρίνονται καλά στην αύξηση των αριθμών χρηστών, ειδών και βαθμολογιών. Επιπλέον, προσφέρουν ένα μοντέλο που μπορεί να αποθηκευτεί αποδοτικά και μπορεί να εκπαιδευτεί σχετικά εύκολα. \par
Στη βασική τους μορφή, μοντελοποιούν τις αλληλεπιδράσεις χρηστών-ειδών μετασχηματίζοντας χρήστες και είδη στον ίδιο χώρο λανθάνοντων παραγόντων {(\en latent factors)}. Kάθε χρήστης και κάθε είδος συσχετίζονται με ένα διάνυσμα των οποίων τα στοιχεία καταγράφουν κατά πόσο ο χρήστης ενδιαφέρεται για τους διάφορους παράγοντες και κατά πόσο το είδος τους διαθέτει. Το εσωτερικό τους γινόμενο απεικονίζει το ενδιαφέρον του χρήστη για τα χαρακτηριστικά του είδους. Στη συνέχεια το μοντέλο εκπαιδεύεται, χρησιμοποιώντας τα διαθέσιμα δεδομένα, για να προβλέψει τις βαθμολογίες χρηστών για είδη που δεν έχουν δει. Τα συστήματα αυτά διευρύνουν την κατηγοριοποίηση των ειδών καθώς είναι σε θέση να εντοπίσουν το ενδιαφέρον ενός χρήστη για είδη που έχουν χαρακτηριστικά που εξάγονται αυτόματα από το σύστημα και επομένως να προτείνει πιο στοχευμένα σε αυτόν είδη. Αλγόριθμοι αυτού του είδους, όπως ο {\en SVD++} \cite{Koren:2008:FMN:1401890.1401944} μπορούν να χρησιμοποιήσουν και άλλα στοιχεία του ιστορικού του χρήστη πέρα από τις βαθμολογίες, αυξάνοντας την ακρίβεια των προβλέψεων. Μέθοδοι αυτής της κατηγορίας μπορούν ακόμα να λάβουν υπόψη τους τις αλλαγές στις προτιμήσεις των χρηστών και στα χαρακτηριστικά των ειδών στη διάρκεια του χρόνου, βελτιώνοντας και άλλο την ποιότητα των προβλέψεων. \cite{ricci2011recommender}
\section{Το πρόβλημα της Κρύας Εκκίνησης} 
Το πρόβλημα της κρύας εκκίνησης αναφέρεται στην συμπεριφορά του συστήματος στην εισαγωγή νέων χρηστών και ειδών και στη δυσκολία παραγωγής αξιόπιστων συστάσεων λόγω της αρχικής έλλειψης βαθμολογιών. Είναι πρόβλημα που αντιμετωπίζουν όλα τα συστήματα παραγωγής συστάσεων, τόσο αυτά που βασίζονται στο περιεχομένο όσο και αυτά που εφαρμόζουν συνεργατική διήθηση. Για τα δεύτερα είναι πιο σοβαρό πρόβλημα καθώς βασίζονται αποκλειστικά στις βαθμολογίες των χρηστών. Μπορεί να θεωρηθεί και ως πρόβλημα μειωμένης κάλυψης του χώρου των ειδών.\cite{ricci2011recommender} Διακρίνεται κυρίως σε τρεις κατηγορίες \cite{Nikolakopoulos2015126}: 
\begin{itemize}
 \item Πρόβλημα Νέας Κοινότητας ({\en New Community Problem}): Στην αρχή της λειτουργίας ενός συστήματος, οι βαθμολογίες είναι αναγκαστικά λίγες. Αυτό οδηγεί αραιά σύνολα δεδομένων και δεν επιτρέπει στα συστήματα συνεργατικής διήθησης να εντοπίσουν τις απαραίτητες συσχετίσεις ανάμεσα σε χρήστες και είδη.
 \item Προβλημα Νέων Χρηστών ({\en New Users Problem}): Το πρόβλημα αυτό εμφανίζεται με την εισαγωγή νέων χρηστών στο σύστημα, οι οποίοι δε μπορούν να λάβουν προσωποιημένες συστάσεις, αφού έχουν ελάχιστες βαθμολογίες.
 \item Πρόβλημα Νέων Ειδών ({\en New Items Problem}): Επειδή τα νέα είδη που εισάγονται στο σύστημα εκ των πραγμάτων δεν έχουν βαθμολογηθεί αρκετά, δε μπορούν να συσχετιστούν με τα υπόλοιπα και επομένως να συμπεριληφθούν στις συστάσεις προς τους χρήστες. 
\end{itemize}